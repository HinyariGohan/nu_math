\begin{problem}
   正方形ABCDとその内部の1点Pがある。
線分AP,BP,CPの長さがそれぞれ7,5,1であるとき,
この正方形の面積を求めよ。
\end{problem}

\begin{mawarikomi}{}{%
\begin{zahyou*}[ul=8mm](-0.5,7)(-0.5,7)
  % 正方形点列定義,線分描画
  \tenretu{A(0,4*sqrt(2))nw;B(0,0)sw;C(4*sqrt(2),0)se;D(4*sqrt(2),4*sqrt(2))ne}
  \Takakkei{\A\B\C\D}
  % 円の交点
  \CandC\A{7}\B{5}\F\P
  % 線分描画
  \Drawline{\P\A}
  \Drawline{\B\P}
  \Drawline{\C\P}
  % 辺弧に大きさを描画
  \HenKo\A\P{$7$}
  \HenKo\P\B{$5$}
  \HenKo<henkoH=0.5zw,henkosep=0.25pt>\P\C{$1$}
  % 点Pにラベル,黒丸
  \tenretu**{P[ne]}
  \Kuromaru<size=2pt>\P
  % PからADへの垂線
  \Suisen\P\A\D\Q
  \HenKo\Q\A{$p$}
  \HenKo\D\Q{$q$}
  \Hasen{\P\Q}
  % PからCDへの垂線
  \Suisen\P\C\D\R
  \HenKo\R\D{$a$}
  \HenKo\C\R{$b$}
  \Hasen{\P\R}
  % PD
  \Drawline{\P\D}
  \HenKo<henkoH=0.4zw,henkosep=0.25pt,putpos=.4>\P\D{$x$}
\end{zahyou*}
}

$\mathrm{PD} = x$とする.正方形内に存在する直角三角形に対し,三平方の定理より,
$\begin{emcases}
  a^2 + q^2 = x^2\\
  a^2 + p^2 = 49 \\
  b^2 + q^2 = 1 \\
  b^2 + p^2 = 25
\end{emcases}\houteisiki{\label{196321}}$
が成り立つ.

\eqref{196321} について,各式より,
$\begin{emcases}
  p^2 - q^2 = 49-x^2\\
  p^2 - q^2 = 24
\end{emcases}$
を得る.よって,これを解いて,$x=5$を得る.

また,$\sankaku{ABP}\godo\sankaku{ADP}$であるから,$\kaku{BAP} = \kaku{DAP} = \bunsuu\pi4$となる.よって,点Pは対角線上に存在し,対角線の長さが8となるから,求める正方形の面積を$S$とすると,
\[S = 8 \times 8 = 64\]
\hfill$\Y S=64$

\end{mawarikomi}
