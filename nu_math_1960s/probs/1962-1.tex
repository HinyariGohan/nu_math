\begin{problem}
   次の各問について空欄に数,式,言葉を記入せよ。
\begin{enumerate}
\item 三角形$ABC$の内部に点$P$があって,
三角形$PBC$,$PCA$,$PAB$の面積の比が,
3辺$BC$,$CA$,$AB$の長さの比に等しいとき,
$P$は三角形$ABC$の(ア)□である。
\item 空間で,定線分$AB$に対し,
$\angle APB$が鈍角になるような点$P$の存在範囲は(イ)□である。
\item 3乗すると8になる複素数は(ウ)□である。
\item $x$の方程式$2^x+2^{-x}=a$に根があるための$a$の値の範囲は(エ)□で,
そのとき$x=$(オ)□である。
\item 次の無限級数に和があるのは(カ)□の場合で,
そのとき,和は(キ)□である。
\[ x+x(1-x)+x{(1-x)}^2+\cdots+x{(1-x)}^n+\cdots \]
\item $\displaystyle\int_0^xf(t)dt=\cos3x+c$($c$は定数)のとき,
$f(t)$は(ク)□であり,$c$は(ケ)□である。
\end{enumerate}

\end{problem}
