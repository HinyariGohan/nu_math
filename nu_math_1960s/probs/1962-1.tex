\begin{problem}
  次の各問について空欄に数,式,言葉を記入せよ。
\begin{enumerate}
\item 三角形$ABC$の内部に点$P$があって,
三角形$PBC$,$PCA$,$PAB$の面積の比が,
3辺$BC$,$CA$,$AB$の長さの比に等しいとき,
$P$は三角形$ABC$の\HAKO である。
\item 空間で,定線分$AB$に対し,
$\angle APB$が鈍角になるような点$P$の存在範囲は\HAKO である。
\item 3乗すると8になる複素数は\HAKO である。
\item $x$の方程式$2^x+2^{-x}=a$に根があるための$a$の値の範囲は\HAKO で,
そのとき$x=$\HAKO である。
\item 次の無限級数に和があるのは\HAKO の場合で,
そのとき,和は\HAKO である。
\[ x+x(1-x)+x{(1-x)}^2+\cdots+x{(1-x)}^n+\cdots \]
\item $\displaystyle\int_0^xf(t)dt=\cos3x+c$($c$は定数)のとき,
$f(t)$は\HAKO であり,$c$は\HAKO である。
\end{enumerate}
\end{problem}

\begin{enumerate}
  \item \fbox{ア}\qquad 内心
  \item \fbox{イ}\qquad AB を直径とする球の内部
  \item ある複素数を$\omega$とする.
  \begin{align*}
    \omega^3 &= 8 \\
    \omega^3-8 &= 0 \\
    (\omega-2)(\omega^2+2\omega+4) &= 0
  \end{align*}
  これを解いて,\fbox{ウ}\qquad $2, -1\pm\sqrt3 i$
  \item 相加相乗平均の公式より,
  $2^x + 2^{-x} \geq 2$ (等号成立は $x=0$のとき) であるから,根があるための$a$の範囲は,\fbox{エ}\qquad $a\geq 2$.

  \begin{align*}
    2^x + 2^{-x} &= a \\
    2^{2x} - a \cdot 2^x + 1 &= 0\\
    2^x &= \frac{a\pm \sqrt{a^2-4}}{2}
  \end{align*}
  したがって,両辺底を2とする対数をとって,\fbox{オ}\qquad $x = \log_2\left( a\pm \sqrt{a^2-4} \right) -1$

  \item $\zettaiti{1-x} < 1$より,\fbox{カ}\qquad $0<x<2$のとき.
  \begin{enumerate}
    \item $x=1$のとき,1
    \item $x\neq1$のとき,$\frac{x}{1-(1-x)} = \frac{x}{x} = 1$
  \end{enumerate}
  よって,(i),(ii)より,求める和は\fbox{キ}\qquad 1である.

  \item $\dint{0}{x}f(t)\,dt = \cos 3x + C$であり,両辺を$x$で微分すると,$f(x) = -3\sin 3x$である.したがって\fbox{ク}\qquad $f(t) = -3 \sin 3x$.

  また,$\dint{0}{x}(-3\sin 3t)\,dt = \teisekibun{\cos 3x}{0}{t} = \cos 3t+1$であるから,\fbox{ケ}\qquad $c=1$である.
\end{enumerate}
