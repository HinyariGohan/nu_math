\begin{center}
  【数学I代数】
\end{center}

\begin{problem}
   A市から,B,C両地を経てD市へいたる電車道路にそって,甲と乙とがAD間を往復する.
  甲は,往きには,ABC間は電車CD間は歩いて,$p$時間,帰りには,DCB間は電車BA間は歩いて,$q$時間を要した.また乙は,往きには,AB間は電車BCD間は歩いて,$r$時間,帰りには,DC間は電車CBA間は歩いて,$s$時間を要した.距離はAB間$a$km,BC間$b$km,CD間$c$kmで,甲,乙の徒歩の速さはそれぞれ一定,電車の速さも一定とすると,$a,b,c,p,q,r,s$の間にどのような等式が成り立つか.ただし,電車を待つ時間は所要時間に含めない.
\end{problem}

\begin{problem}
   $a,b,c$は正の定数とする.このとき,
  \begin{enumerate}
    \item 正数$x,y$の和が一定数$h$に等しいならば,
    $\frac{a^2}{x} + \frac  {b^2}{y} \geq \frac{(a+b)^2}{h}$
    となることを証明せよ.
    \item 上の結果を使って,正数$x,y,z$の和が一定数$k$に等しいとき,
    $\frac{a^2}{x} + \frac{b^2}{y} + \frac{c^2}{z}$の最小値を求めよ.
  \end{enumerate}
\end{problem}

\resetcounter{daimon}

\begin{center}
  【数学I幾何】
\end{center}

\begin{problem}
   定円周上を同一方向に同じ速さで二つの点P,Qがまわっている.円周上に二定点A,Bをとって,二直線AP,BQを作れば,そのなす角は一定に保たれることを証明せよ.
\end{problem}

\begin{problem}
   三角形ABCで$\mathrm{BC} = a,\mathrm{CA} = b, \mathrm{AB} = c\ (b<c)$とする.辺BC上に点M,D,Hがあって,AMは中線,ADは頂角Aの二等分線,AHはBCへの垂線とするとき,線分BD,BHの長さ,および
  $\frac{\mathrm{MD}}{\mathrm{MH}}$を$a,b,c$で表せ.
\end{problem}

\resetcounter{daimon}

\begin{center}
  【数学II】
\end{center}

\begin{problem}
   \ten{}{x}{y}を座標とする点Pが,原点を中心とする半径1の円周上を一様な速さでまわっている.このとき,$u = x(x+y), v = y(x+y)$なる関係で定まる\ten{}{u}{v}を座標とするQも,ある円の周上を一様な速さでまわっていることを証明せよ.
\end{problem}
\begin{problem}
   放物線$y=x^2$の上に相異なる三点A,B,Cがある.このとき,
  \begin{enumerate}
    \item 直線ABが放物線$y = \bunsuu14 x^2+ 1$に接するための条件を,点A,Bの$x$座標の間の関係式で表せ.
    \item 二直線AB,BCが放物線$y = \bunsuu14x^2 + 1$に接するならば,直線CAもまたこの放物線に接することを証明せよ.
  \end{enumerate}
\end{problem}

\resetcounter{daimon}

\begin{center}
  【数学III】
\end{center}
\begin{problem}
   数列$c_1,c_2,\cdots,c_n,\cdots$において,一般項$c_n$が公差$d$なるある等差数列の第$n$項と公比$r$なるある等比数列の第$n$項との和として表されるためには,
  \[c_{n+2} - (1+r)c_{n+1} + rc_n = d(1-r)\quad (n=1,2,\cdots)\]
  が成り立つことが必要かつ十分である.これを証明せよ.
\end{problem}
\begin{problem}
   $x$軸および$y$軸の上に両端を置いて第一象限内を動く長さ1の線分がある.
  \begin{enumerate}
    \item この線分と定直線$x=k$との交点の中で$y$座標が最大になる点をPとする.Pの$y$座標を求めよ.
    \item $k$が0から1まで変わるとき,点Pの動いて出来る線の方程式を求め,かつ,この線の略図をかけ.
  \end{enumerate}
\end{problem}
