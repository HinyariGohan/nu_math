
\begin{problem}
  三角形$ABC$の3辺$BC$,$CA$,$AB$の長さをそれぞれ$a$,$b$,$c$とするとき,
$\angle A=2\angle C$ならば$a^2=c(b+c)$であることを証明せよ。
\end{problem}

$\kaku{C} = \theta$とする.正弦定理から,$k$を実数として,
\[\frac{a}{\sin 2\theta} = \frac{b}{\sin3\theta} = \frac{c}{\sin \theta} = k\]
を満たす.したがって,$a = k\sin 2\theta, b = k\sin 3\theta, c= k\sin\theta$である.

\begin{align*}
  (左辺) &= k^2\sin^22\theta \\
  &= 4k^2 \sin^2\theta \cos^2\theta \\
\end{align*}
\begin{align*}
  (右辺) &= k^2 \sin \theta(\sin 3\theta + \sin \theta) \\
  &= k^2 \sin \theta(3\sin\theta-4\sin^3\theta + \sin\theta) \\
  &= -4k^2(\sin 4\theta-\sin^2\theta) \\
  &= -4k^2\sin^2\theta(\sin^2\theta-1) \\
  &= 4k^2\sin^2\cos^2\theta
\end{align*}
より,$(左辺) = (右辺)$となるため,$\kaku{A} = 2\kaku{C}$であれば,$a^2 = c(b+c)$となる.\qed
