\begin{problem}
$\sqrt{x-a}=x-b$を$x$について解け。
\end{problem}

定義域から$x-a>0$より,$x>a$の範囲で以下を考える.
\begin{enumerate}[(i)]
  \item $x-b>0$のとき

  両辺を二乗すると
  \begin{align*}
    x-a &= (x-b)^2 \\
    x^2 - (2b+1)x + b^2 + a &= 0
  \end{align*}
  ここで,この2次方程式に対して判別式を$D$とすると,
  \begin{align*}
    D &= (2b+1)^2 - 4(b^2+a) \\
    &= 4(b-a)+1
  \end{align*}
  $D \geq 0$のとき,$x = \frac{(2b-1)\pm\sqrt{4(b-a)+1}}{2}$である.
  ここで,条件より$x>b$かつ$x>a$であるから,$x = \frac{(2b-1) + \sqrt{4(b-a)+1}}{2}$である.

  \item $x-b<0$のとき

  $y = \sqrt{x-a}$,$y = x-b$について考えると,$\sqrt{x-a} > 0$かつ,$x-b<0$となるため,明らかに解を持たない.
\end{enumerate}

(i),(ii)より,解は以下の通りとなる.

\Ans{$x = \trenritu{%
  \frac{(2b-1) + \sqrt{4(b-a)+1}}{2} & (4(b-a) + 1 \geq 0\ のとき) \\
  解なし & (4(b-a) + 1 < 0\ のとき)
  }$
}
