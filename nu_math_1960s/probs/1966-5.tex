\begin{problem}
1次式$f_n(x)$ $(n=1, \, 2, \, 3, \cdots\cdots)$が
$f_1(x)=1+x,\ x^2f_{n+1}(x)=x^2+x^3+\int_0^xtf_n(t)dt$
を満たしているとき
\begin{enumerate}
\item $f_n(x)$ $(n=1, \, 2, \, 3, \, \cdots\cdots)$を求めよ.
\item $\displaystyle\lim_{n\to\infty}f_n(1)$を求めよ.
\end{enumerate}
\end{problem}


\begin{enumerate}
    \item $f_n(x) = ax+ b$と表されることを数学的帰納法を用いて示す.
    \begin{enumerate}[(i)]
        \item
        $n=1$のとき\quad $f_1(x) = 1 + x$より成立.
        \item
        $n=k$のとき\quad $f_k(x) = ax + b$とする.このとき,
\begin{align*}
    x^2 f_{k+1}(x) &= x^3 + x^2 + \dint{0}{x} t(at+b)\,dx\\
    x^2 f_{k+1}(x) &= x^3 + x^2 + \teisekibun{\bunsuu12at^3 + \bunsuu12bt^2}{0}{x}\\
    f_{k+1}(x) &= \left(1 + \bunsuu{a}{3}\right) + \left(1 + \bunsuu{b}{2}\right)
\end{align*}
より,$n = k + 1$のときも成立.
\end{enumerate}

(i),(ii)より,すべての自然数$n$について,$f_n(x)  = ax + b$と表される.

よって,$f_n(x) = a_nx + b_n$とすると,
\[ f_{n+1}(x) = \left(1 + \bunsuu{a_n}{3}\right)x + \left(1 + \bunsuu{b_n}{2}\right) \]
より,

\begin{numcases}{}
    a_{n+1} = 1 + \bunsuu{a_n}{3} \label{196651} \\
    b_{n+1} = 1 + \bunsuu{b_n}{2} \label{196652}
\end{numcases}
である.

\eqref{196651}から,
\begin{align*}
    a_{n+1} - \bunsuu32 &= \bunsuu13 \left(a_n - \bunsuu32\right)\\
    a_n - \bunsuu32 &= - \bunsuu12 \left(\bunsuu13\right)^{n-1}\\
    a_n &= - \bunsuu12 \left(\bunsuu13\right)^{n-1} + \bunsuu32
\end{align*}

\eqref{196652}から,
\begin{align*}
    b_{n+1} - 2 &= \bunsuu12 (b_n -2)\\
    b_n -2 &= -\left(\bunsuu12\right)^{n-1}\\
    b_n &= 2 - \left(\bunsuu12\right)^{n-1}
\end{align*}

よって,\Ans{$f_n(x) = \left\{-\bunsuu12 \left(\bunsuu13\right)^{n-1} + \bunsuu32 \right\}x + 2 - \left(\bunsuu12\right)^{n-1}$}

\item
\begin{align*}
\dlim{n \to \infty} f_n(1) &= \left\{-\bunsuu12\left(\bunsuu13\right)^{n-1} + \bunsuu32\right\} + 2 - \left(\bunsuu12\right)^{n-1} \\
&=\bunsuu32 + 2\\
&=\bunsuu72
\end{align*}
\end{enumerate}
