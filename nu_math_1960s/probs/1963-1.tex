\begin{problem}
    3辺が$BC>CA>AB$となるような三角形$ABC$がつくれるためには,
頂角$B$の大きさがどんな範囲にあることが必要かつ十分か。
\end{problem}

$\mathrm{AB} = c, \mathrm{BC} = a, \mathrm{CA} = b$とする.

このとき,正弦定理より,
\[\bunsuu{a}{\sin A} = \bunsuu{b}{\sin B} = \bunsuu{c}{\sin C} = k\]
が成立する.また,$\mathrm{BC} > \mathrm{CA} > \mathrm{AB}$から,

\[\sin A > \sin B > \sin C\]

となるようにすると,

\[\mathrm{A} + \mathrm{B} + \mathrm{C} = \bunsuu{\pi}2\]

ここで,
\[\sin\left(\bunsuu{\pi}{4} - x\right) = \sin\left(\bunsuu{\pi}{4} + x\right) \qquad \left(0 < x < \bunsuu{\pi}{4} のとき\right)\]

\[\mathrm{A} > \mathrm{B} > \mathrm{C}\]

であることがわかる.

したがって,$0 < \mathrm{B} < \bunsuu{\pi}{4}$
