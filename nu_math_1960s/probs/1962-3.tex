\begin{problem}
   $f(x)=x^4+2x^3+ax$が極大値をもつのは,
定数$a$がどんな範囲にある場合か。
\end{problem}

$f(x) = x^4 + 2x^3 + ax$とする.このとき,$f'(x) = 4x^3 + 6x^2 + a, f''(x) = 12x^2 + 12x$である.

増減表は以下の通りになる.

\RESETKEYA
\setkeys{zogen}{
  hensu=x,
  ranab=-1,ranad=0,ranba=+,ranbb=0,ranbc=-,ranbd=0,ranbe=+,
  ranca=\nearrow,rancb=a+2,rancc=\searrow,rancd=a,rance=\nearrow
}
\zogen(3,5)

ここで,$f(x)$が極大値をもつためには,$f'(x)$の値が正から負に変わればよい.

\begin{enumerate}[(i)\ ]
  \item $a>0$のとき,
  \RESETKEYA
  \setkeys{zogen}{%
  hensu=x,
  ranad=-1,ranaf=0,
  ranba=-,ranbb=0,ranbc=+,ranbd=a+2,ranbe=+,ranbf=a,ranbg=+,
  ranca=\searrow,rance=\nearrow,rancg=\nearrow}
  \zogen(3,7)
  より不適.


  \item $a=0$のとき,
  \RESETKEYA
  \setkeys{zogen}{%
    hensu=x,
    ranad=-1,ranaf=0,
    ranba=-,ranbb=0,ranbc=+,ranbd=2,ranbe=+,ranbf=0,ranbg=+,
    ranca=\searrow,rance=\nearrow,rancg=\nearrow}
  \zogen(3,7)
  より不適.

  \item $-2<a<0$のとき,
  \RESETKEYA
  \setkeys{zogen}{
    hensu=x,
    ranad=-1,ranag=0,
    ranba=-,ranbb=0,ranbc=+,ranbd=a+2,ranbe=+,ranbf=-,ranbg=a,ranbh=-,
    ranbi=0,ranbj=+,
    ranca=\searrow,rance=\nearrow,rancf=\searrow,rancj=\nearrow
  }
  \zogen(3,10)
  より,$f(-1) = a+2 > 0, f(0) = a < 0$であることと,中間値の定理より,$f'(x)$は$-1<x<0$の範囲で,少なくとも1回は正から負への符号変化をする.

  \item $a = -2$のとき,
  \RESETKEYA
  \setkeys{zogen}{%
    hensu=x,
    ranab=-1,ranad=0,
    ranba=-,ranbb=0,ranbc=-,ranbd=-2,ranbe=-,ranbf=0,ranbg=+,
    ranca=\searrow,rancc=\searrow,rance=\searrow,rancg=\nearrow}
  \zogen(3,7)
  より不適.

  \item $a < -2$のとき,
  \RESETKEYA
  \setkeys{zogen}{%
    hensu=x,
    ranab=-1,ranad=0,
    ranba=-,ranbb=a+2,ranbc=-,ranbd=a,ranbe=-,ranbf=0,ranbg=+,
    ranca=\searrow,rancc=\searrow,rance=\searrow,rancg=\nearrow}
  \zogen(3,7)
  より不適.
\end{enumerate}

(i)~(v)より,$-2 < a < 0$のとき極大値をもつ.
