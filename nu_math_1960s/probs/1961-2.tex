\begin{problem}
    \begin{enumerate}
\item 平面上で座標の原点を$O$とし,第1象限内にある点$A$の座標を$(a, \, b)$とする。
$A$を通って$x$軸の正の部分,$y$軸の正の部分とそれぞれ$P$,$Q$で交わる直線を引いて,
三角形$OPQ$の面積が1になるようにするには,
直線$PQ$の傾き(勾配)をどのような値にとればよいか。
また,このようなことが可能であるための点$A$の存在範囲を求めて,
これを図示せよ。
\item ある試験で$A$組と$B$組の全員が受験し,
その成績の平均点を少数第2位以下を四捨五入して求めたところ,
全体では73.2,$A$組では70.5,$B$組では75.6であった。
$A$,$B$ 2組の人数の合計が100人ならば,$A$組の人数はどのような範囲にあるか。
\end{enumerate}
\end{problem}

\begin{enumerate}
    \item PQの直線の傾きを$m$として,$\ell :y = m(x-a) + b$とおく.
    (ただし,条件より $m<0$)

    このとき,P,Q の座標はそれぞれ,
    \tenbig{P}{a-\bunsuu{b}{m}}{0},
    \ten{Q}{0}{-am+b}
    となる.

    このとき,\sankaku{OPQ} の面積$S$は,
    \begin{align*}
        S &= \bunsuu{1}{2}\left(a = \bunsuu{b}{m}\right)(-am + b)\\
        &=\bunsuu12 \left(-a^2m + ab + ab - \bunsuu{b^2}{m}\right)\\
        &=-\bunsuu{1}{2m} (a^2m^2-2abm+b^2)\\
        &=-\bunsuu{1}{2m} (am-b)^2 = 1
    \end{align*}

    よって,$\houteisiki{(am-b)^2=-2m} \label{196121}$

    傾き$m$は,
    \[(am-b)^2 = -2m \quad (m<0)\]
    を満たせばよい.

    $\eqref{196121} \doti a^2m^2-2abm+2m+b^2 = 0$\quad
    であり,$m<0$において,この式が成り立てばよいので,
    \[m:y=a^2m^2-2(ab-1)m + b^2\]
    とおくと,
    \begin{align}
        (mの軸) &< 0 \label{196122}\\
        y_{lm} &> 0 \label{196123}\\
        (y = 0 についての判別式D) &\geq 0 \label{196124}
    \end{align}
    であればよい.

    \eqref{196122}より,$ab-1<0 \doti ab < 1$

    \eqref{196123}より,$b^2 > 0$

    \eqref{196124}より,
    \begin{align*}
        D &= (ab-1)^2-a^2b^2\\
        &=-2ab + 1\geq 0
        &2ab \leq 1 \Y ab \leq \bunsuu12
    \end{align*}

    よって,上記より求める点Aの存在範囲は下図の斜線部分
    (ただし$x$軸,$y$軸の境界を除く)

    \begin{center}
        \begin{pszahyou}[ul=12mm](-0.5,3)(-0.5,3)
            \def\Fx{1/(2*X)}
            \def\Gx{0}
            \YNurii*\Fx\Gx{0.01}{3}
            \YGurafu*\Fx
        \end{pszahyou}
    \end{center}
    \KOTAE


    \item 全体の平均値を$U$,A組の平均値を$A$,B組の平均値を$B$とおく.ここで,A組の人数を$a$とおくと,
    \begin{align}
        U = \bunsuu{a \cdot A + (100-a) \cdot B}{100} \label{196121}
    \end{align}

    条件より,\begin{align*}
        73.15 \leq U < 73.25 \\
        70.45 \leq A < 70.55 \\
        75.55 \leq B < 75.65
    \end{align*}

    \begin{align}
        \eqref{196121} &\doti 100U = a(A-B) + 100B \\
        &\doti a = 100 \bunsuu{U-B}{A-B} \label{196122}\\
        & =100 \left(1 + \bunsuu{U-A}{A-B}  \label{196123}\right)
    \end{align}

    \begin{enumerate}
        \item $a$が最小値をとるときを考える.

        $B$を固定したとき$a$が最小となるのは,$U$が最大であり,$A$が最小のときである.
        $(\nazenara \eqref{196122})$

        このとき,$U - A < 2.8, -5.2 < A-B < -5.1$である.

        よって,$-\bunsuu{2.8}{5.1} < \bunsuu{U-A}{A-B}$

        したがって,\begin{align*}
            a &> 100\left(1-\bunsuu{2.8}{5.1}\right)\\
            &=100 \cdot \bunsuu{13-4}{13}\\
            &=100 \cdot \bunsuu{9}{13}\\
            &=69.2\ldots
        \end{align*}

        \item $a$が最大値をとるときを考える.

        $B$を固定したとき$a$が最大値をとるには,$U$が最小,$A$が最大の時である.
        $(\nazenara \eqref{196122})$

        このとき,$U - A > 2.6, -5.1 < A - B < -5$である.

        よって,$\bunsuu{U-A}{A-B} < -\bunsuu{2.6}{5.1}$

        したがって,\begin{align*}
            a &< 100\left(1-\bunsuu{2.6}{5.1}\right)\\
            &= 100\left(\bunsuu{7-2}{7}\right) \\
            &= 100 \cdot \bunsuu57 = 71.4\ldots
        \end{align*}
    \end{enumerate}

    (i), (ii) より,$a= 70.71$
    \Ans{$a = 70.71$}
\end{enumerate}
