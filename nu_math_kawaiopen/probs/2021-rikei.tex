\begin{problem}
$a$を正の実数とする.関数$f(x) = x^2 e^{-ax}$について,以下の問に答えよ.必要ならば,任意の自然数$n$に対して,$\dlim{x \to \infty} x^n e^{-x} = 0$が成り立つことを用いて良い.
\begin{enumerate}
  \item $f(x)$の極値を求めよ.
  \item $b$を実数とする.関数$g(x) = f(x) - bx$が正の極小値をもつような$b$の範囲を$a$を用いて表せ.
\end{enumerate}
\end{problem}

\begin{problem}
$n$を2以上の整数とし,整数$x,y$が
\[ 5x + 2y = 7^n \qquad \cdots (*)\]
を満たしている.以下の問に答えよ.
\begin{enumerate}
  \item (*)を満たす整数の組$(x,y)$をすべて求めよ.
  \item (*)を満たす整数$x,y$について,$x$と$y$が互いに素でないならば,$x,y$がともに7の倍数であることを示せ.
  \item (*)かつ$x \geq 0, y \geq 0$を満たす整数の組$(x,y)$のうち,$x$と$y$が互いに素であるものの個数を$p_n$とする.$\dlim{n \to \infty}\ \bunsuu{p_n}{7^n}$を求めよ.
\end{enumerate}
\end{problem}

\begin{problem}
$xy$平面上で原点Oを中心とする半径1の円を$C$とし,$x$軸上に2点\ten{A}{a}{0},\ten{B}{3}{0}をとる.Bを通り傾き$m$の直線を$l$とし,$l$は$C$と異なる2点P,Qで交わっている.以下の問に答えよ.
\begin{enumerate}
  \item $m$のとり得る値の範囲を求めよ.
  \item 2点P,Qの$x$座標をそれぞれ$p,q$とするとき,$p+q$および$pq$を$m$を用いて表せ.
  \item 積$\mathrm{AP} \cdot \mathrm{AQ}$を$a$と$m$を用いて表せ.
  \item (3)の$\mathrm{AP} \cdot \mathrm{AQ}$が$m$の値によらず一定となるような$a$をすべて求めよ.
\end{enumerate}
\end{problem}

\begin{problem}
$n$を3以上の整数として,番号$1,2,\cdots,n$の人がいる.番号1の人は白玉1個と赤玉1個,番号2の人は白玉1個と青玉1個,番号$3,4,\cdots,n$の人は白玉2個を持っている.

\quad 最初に,番号1の人は,持っている2個の玉から無作為に1個選んで番号2の人に渡す.次に,番号2の人は,番号1の人から受け取った玉を含む3個の玉から無作為に1個選んで番号3の人に渡す.このようにして順番に,番号$i(i=2,3,\cdots,n-1)$の人は,番号$i-1$の人から受け取った玉を含む3個の玉から無作為に1個選んで番号$i+1$の人に渡す.最後に番号$n$の人は,番号$n-1$の人から受け取った玉を含む3個の玉から無作為に1個選んで番号1の人に渡す.これを1回の操作とする.1回目の操作を終えた時点で,赤玉と青玉の両方を持っている人がいるときはその人を勝者とする.勝者が決まらなかったときは,1回目の操作後に確認が持っている玉の状態から2回目の操作を行い,2回目の操作を終えた時点で赤玉と青玉の両方を持っている人がいるときはその人を勝者とする.なお,1回目の操作で勝者が決まったときは2回目の操作を行わない.以下の問に答えよ.
\begin{enumerate}
  \item 1回目の操作で番号$k\ (k=1,2,\cdots,n)$の人が勝者となる確率を$p_n$とする.$p_1,p_2$および$p_k\ (k=3,4,\cdots,n)$を求めよ.
  \item 2回目の操作で番号$k\ (k=1,2,\cdots,n)$の人が勝者となる確率を$q_k$とする.$q_1,q_2$を求めよ.
  \item 2回目までの操作で勝者が決まる確率を$r_n$とする.$\dlim{n \to \infty}r_n$を求めよ.必要ならば,$\dlim{n \to \infty} \bunsuu{n}{3^n}=0$であることを用いてよい.
\end{enumerate}
\end{problem}
