\begin{problem}
実数を係数とする3次方程式
  \begin{center}
    $x^3+px^2+qx+r=0$
  \end{center}
は,相異なる虚数解$\alpha,\beta$と実数解$\gamma$をもつとする.
  \begin{enumerate}
    \item $\beta = \overline{\alpha}$ が成り立つことを証明せよ.ここで,$\overline{\alpha}$は$\alpha$と共役な複素数を表す.
    \item $\alpha,\beta,\gamma$が等式$\alpha\beta+\beta\gamma+\gamma\alpha=3$を満たし,さらに複素数平面上で$\alpha,\beta,\gamma$を表す3点は1辺の長さが$\sqrt{3}$の正三角形をなすものとする.このとき,実数の組$(p,q,r)$をすべて求めよ.
  \end{enumerate}
\end{problem}
\kaie %(2)図表作成%
\begin{enumerate}
  \item
  \begin{align}
    x^3+px^2+qx+r=0
  \end{align}
  とする. $\alpha$は①の解の一つであるから,
  \begin{align*}
    \alpha^3+p\alpha^2+q\alpha+r &= 0\\
    両辺に共役な複素数をとると,   \overline{\alpha^3+p\alpha^2+q\alpha+r} &= \overline{0}\\
    \overline{\alpha^3}+p\overline{\alpha^2}+q\overline{\alpha}+\gamma &= 0\\
  \end{align*}
  より,$\overline{\alpha} \neq \alpha$. いま,$\alpha$と$\beta$は①の虚数解の2つであるから,$\overline{\alpha}$は$\alpha$と共役な複素数であることを利用すると,$\beta = \overline{\alpha}$といえる.\\
  \item 複素数平面上で$\alpha,\beta,\gamma$を表す3点が表す図形は正三角形であるから,%
  \begin{enumerate}[(i)\ ]
    \item $\alpha,\beta$の実数部分が$\gamma$よりも大きいとき
    \begin{align*}
      \gamma=\alpha とすると,   \alpha &= a+\bunsuu32+\bunsuu{\sqrt3}2i\\
                                \beta &= a+\bunsuu32-\bunsuu{\sqrt3}2i  とかける.
    \end{align*}
    \begin{align*}
      \alpha\beta+\beta\gamma+\gamma\alpha &= \left(a+\bunsuu32\right)^2+\bunsuu34+2a\left(a+\bunsuu32\right)\\
      &= a^2+3a+3+2a^2+3a\\
      &= 3a^2+6a+3\\
      &= 3(a+1)^2 =3  より,a=0,-2となる.
    \end{align*}
    \begin{enumerate}[(A)\ ]
      \item $a=0$のとき\\
      $\left\{%
          \begin{array}{l}
            \alpha+\beta+\gamma=-p\\
            \alpha\beta+\beta\gamma+\gamma\alpha=q
            \alpha\beta\gamma=-r
          \end{array}
      \right.$   より,
      \begin{center}
        $p=-3,q=3,r=0$
      \end{center}
      \item $a=-2$のとき
      \begin{center}
        同様にして,  $p=3,q=3,r=2$\\
        \vspace{15pt} $\therefore$  \emkasen<kasensyu=wave>{$(p,q,r)=(-3,3,0),(3,3,2)$}
      \end{center}
    \end{enumerate}
  \item $\alpha,\beta$の実数部分が$\gamma$よりも小さいとき
  \begin{align*}
      \gamma=\alpha とすると,   \alpha &= a-\bunsuu32+\bunsuu{\sqrt3}2i\\
                                \beta &= a-\bunsuu32-\bunsuu{\sqrt3}2i  とかける.
    \end{align*}
    \begin{align*}
      \alpha\beta+\beta\gamma+\gamma\alpha &= \left(a-\bunsuu32\right)^2+\bunsuu34+a(2a-3)\\
      &= a^2-3a+3+2a^2-3a\\
      &= 3a^2-6a+3\\
      &= 3(a-1)^2 =3  より,a=0,2となる.
    \end{align*}
    \begin{enumerate}[(A)\ ]
      \item $a=0$のとき
      \begin{center}
        (i)のときと同様にして,  $p=3,q=3,r=0$
      \end{center}
      \item $a=2$のとき
      \begin{center}
        同様にして,  $p=-3,q=3,r=-2$\\
        \vspace{15pt} $\therefore$  \emkasen<kasensyu=wave>{$(p,q,r)=(3,3,0),(-3,3,-2)$}
      \end{center}
    \end{enumerate}
  \end{enumerate}
  \vspace{15pt}
  よって,(i)(ii)から,
  \Ans{$(p,q,r)=(-3,3,0),(3,3,2),(3,3,0),(-3,3,-2)$}
\end{enumerate}
