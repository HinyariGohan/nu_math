\begin{problem}
    \quad $a$を実数とし,関数$f(x)=x^2+ax-a$ と $F(x)=\displaystyle\int^x_0 f(t)dt$を考える.\\
    \quad 関数 $y=F(x)-f(x)$ のグラフが $x$ 軸と異なる3点で交わるための $a$ の条件を求めよ.
\end{problem}

\kaie % align* 環境は自動で新しい行に挿入されるので「 \\ 」はなくてもいい
\begin{align*}
    F(x) &= \int^x_0(t^2+at-a)dt \\
    &= \left[\bunsuu{1}{3}t^3+\bunsuu{a}{2}t^2-at \right]^x_0 \\
    &= \bunsuu{1}{3}x^3+\bunsuu{a}{2}x^2-ax %「.」はたぶんいらない
\end{align*}

であるので,$g(x)=F(x)-f(x)$ とおくと, % 上と同様に
\begin{align*}
    g(x) &=\left(\bunsuu{1}{3}x^3+\bunsuu{a}{2}x^2 \right) - (x^2+ax-a)\\
    &=\bunsuu{1}{3}x^3+\left(\bunsuu{a}{2}-1\right)x^2-2ax+a\\\\
    g'(x) &= x^2+(a-2)x-2ax\\
    &= (x+a)(x-2)
\end{align*}

$g(x)$は3次関数であるから,$y=g(x)$のグラフが$x$軸と相異なる3点で交わる条件は,
\emkasen{「$g(x)$が極値をもち,かつ,(極大値)$\times$(極小値) $< 0$」}をみたすことである.
% $$ ← 数式モード内で日本語を使うのは,慣習的に避けた方がよいとされる(別にいいと思うけど)
% なので,(極大値) $\times$ (極小値) $<0$ にした.ただし可読性が下がる
% あと,数式モード内で,乗算記号の「×」を使うときは $\times$ を用いる.
これは,\emkasen{「$-a \neq 2,かつ,g(-a)g(2) < 0$」}と言い換えられる.
%\emkasen{} で行をまたぐ下線をエラーなく挿入できる.

% 文のまとまりの間に,1行空行を挿入すると,段落が変わったことになり,自動で先頭に空白が入る.
いま, 
\begin{align*}
    g(-a) &= -\bunsuu{a^3}{3}+\left(\bunsuu{a}{2}-1\right)a^2+2a^2+a\\
    &= \bunsuu{1}{6}a(a^2+6a+6)\\\\
    g(2) &= \bunsuu{8}{3}+2(a-2)-4a+a\\
    &= -a-\bunsuu{4}{3}
\end{align*}
であるので,
$g(-a)g(2) < 0$ より,
\begin{align*}
    \bunsuu{1}{6}a(a^2+6a+6)\left(-a-\bunsuu{4}{3}\right) &< 0\\
    a(a^2+6a+6)\left(a+\bunsuu{4}{3}\right) &> 0
     が求まる.
\end{align*}\\

$\therefore$求める$a$の条件は,
\Ans{$a < -3-\sqrt{3}, -\bunsuu{4}{3} < a < -3+\sqrt{3}, 0 < a$}
%\Ans{答え} で,(答)[ゴシック体]二重下線で答案が出るようにしてある

\hfill % 横いっぱいの空白,縦のときは \vfill (vertical)
(これは,$a = 2$を満たしている.)
