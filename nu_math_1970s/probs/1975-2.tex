\begin{problem}
   次の(1),(2)を解答せよ.
\begin{enumerate}
\item 1から10までの10個の整数から相異なる5個をとり,
その積を$a$,残りの5個の積を$b$とする。
$a \neq b$を証明せよ.
\item また,1から10までの10個の整数のうちの相異なる5個の積として表される整数のうちで,
$\sqrt{10 \, !}$より小さいものの個数を$p$,
$\sqrt{10 \, !}$より大きいものの個数を$q$とする.
$p=q$を証明せよ.
\end{enumerate}
\end{problem}

\begin{enumerate}
  \item 1~10までの10個の整数のうち,7の倍数を含むものは
  7のみだから,$a$または$b$のどちらか一方は7の倍数となるが,もう一方は7の倍数とはならないため,$a \neq b$となる.

  \item 1~10までの10個の整数から5個を選び,その積を$c$,残りの5個の積を$d$とする.ここで,対称性から$c < d$としても一般性を失わない.このとき,$c\cdot d = 10\,!$である.

  ここで,$c < d$から,$c^2 < 10\,! < d^2$となる.よって,$c < \sqrt{10\,!} < d$と表すことができるため,$c$は$\sqrt{10\,!}$よりも小さく,$d$は$\sqrt{10\,!}$よりも大きいことがわかる.

  ここで,$c$の個数と$d$の個数は一致するため,$p = q$となる.
  \hfill
  (証明終了)
\end{enumerate}
