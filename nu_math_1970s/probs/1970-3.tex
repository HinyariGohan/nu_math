\begin{problem}
座標平面で点$P$は$x$軸上を正の方向へ,
点$Q$は$y$軸上を正の方向へ,
点$R$は傾き(勾配)1の直線上を上方へ,
それぞれ一定の速さ$a$,$b$,$c$で動いている.
3点$P$,$Q$,$R$はつねに一直線上にあり,
ある時刻に$P$の位置は$(4, \, 0)$,$Q$の位置は$(0, \, 2)$,$R$の位置は$(2, \, 1)$であった.
このとき$a$,$b$,$c$の値の比を求めよ.
\end{problem}

P,Q,Rが$x$軸上で一直線になるときを$t_1$とする.
このとき,\ten{P}{x}{0},\ten{Q}{0}{0},\ten{R}{1}{0} である.

P,Q,Rが$y$軸上で一直線になるときを$t_0$とする.
このとき,\ten{P}{0}{0},\ten{Q}{0}{y},\ten{R}{0}{-1} である.

また,\ten{P}{4}{0},\ten{Q}{0}{2},\ten{R}{2}{1} となるときを$t_2$とする.

P\qquad $t_0$から$t_2$における距離の変化は,4\houteisiki{\label{19703-1}}

Q\qquad $t_1$から$t_2$における距離の変化は,2\houteisiki{\label{19703-2}}

R\qquad $t_0$から$t_2$における距離の変化は,$2\sqrt2$\houteisiki{\label{19703-3}}

R\qquad $t_1$から$t_2$における距離の変化は,$\sqrt2$\houteisiki{\label{19703-4}}

\eqref{19703-1},\eqref{19703-3}から,PとRの速度の比は,$a:c=4:2\sqrt2$である.

\eqref{19703-2},\eqref{19703-4}から,QとRの速度の比は,$b:c=2:\sqrt2$である.

したがって,$a:b:c=\sqrt2:\sqrt2:1$となる.
