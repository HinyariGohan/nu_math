\begin{problem}
 $f(x)$,$g(x)$を$x\geqq0$で定義された正の値をとる連続関数で,$g(x)$は増加関数であるとする.
このとき
\[ S(x)=\int_0^xf(t)dt, \hspace{1zw} T(x)=\int_0^xf(t)g(t)dt \]
に対して次の(1),(2)を証明せよ.
\begin{enumerate}
\item すべての$x>0$に対して$T(x) \leqq g(x)S(x)$である.
\item $\displaystyle\frac{T(x)}{S(x)}$は$x>0$で増加関数である.
ここで一般に関数$h(x)$が増加関数であるとは,
$x_1<x_2$ならば$h(x_1) \leqq h(x_2)$が成立することをいう.
\end{enumerate}
\end{problem}

\noindent\kaie
\begin{enumerate}
  \item $f(x) = g(x)S(x)-T(x)$とする.
  \begin{align*}
    f(x) &= g(x)S(x) - T(x) \\
    f(x) &= g(x)\dint{0}{x}f(t)\,dt - \dint{0}{x}g(t)f(t)\,dt\\
    f'(x) &= g'(x)\dint{0}{x}f(t)\,dt + g(x)f(x) - g(x)f(x)\\
    &= g'(x)\dint{0}{x}f(t)\,dt
  \end{align*}
  ここで,$g(x)$は増加関数より,$g'(x) > 0$であり,$f(x)$は正の値を取るから,
  $\dint{0}{x}f(t)\,dt > 0$である.したがって,$f'(x) =g'(x)\intd{0}{x}{f}{t} > 0$
  % 増減表省略

  よって,$x>0$において,$g(x)S(x) - T(x) \geq 0$であるから,$g(x)S(x) \geq T(x)$が示された.

  \item
  \begin{align*}
    \bunsuu{T(x)}{S(x)}\,dx &= \bunsuu{T'(x)S(x) - T(x)S'(x)}{S^2(x)} \\
    &= \bunsuu{f(x)g(x)\intd{0}{x}{f}{t} - f(x)\dint{0}{x}f(t)g(t)\,dt}{S^2(x)}\\
    &= \bunsuu{f(x)g(x)\intd{0}{x}{f}{t} - \dint{0}{x}f(t)g(t)\,dt}{S^2(x)} \\
    &= \bunsuu{f(x)\{g(x)S(x)-T(x)\}}{S^2(x)}
  \end{align*}

  $f(x) > 0$かつ(1)から,$g(x)S(x)-T(x) \geq 0$より,$\bunsuu{T(x)}{S(x)}\,dx \geq 0$であるから,$\bunsuu{T(x)}{S(x)}$は増加関数となる.\qed
\end{enumerate}
