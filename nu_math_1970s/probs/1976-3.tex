
\begin{problem}
  自然数$k$および$k$より大きい自然数$n$が与えられているとき,
$1\leqq a_1<a_2<\cdots\cdots<a_k \leqq n$であるような
$k$個の自然数$a_1, \, a_2, \, \cdots\cdots, a_k$の和として表される
自然数は全部で何個あるか.
\end{problem}

$a_1 = 1, a_2 = 2, \cdots, a_k = k$のとき最小値であり,この状態から,
$a_k = k+1, a_k = k+2, \cdots, a_k = n$になるような操作を繰り返す.
次に,$a_{k-1} = k, a_{k-1} = k+1 \cdots, a_k = n-1$のように,以下同様にして,
$a_1 = n - k + 1$になるまですべての値を取ることができる.

よって,最小値は,$\retuwa{l=1}{k}l = \frac12 k(k+1)$であり,
最大値は,$\retuwa{l=1}{k}(n-l+1) = (n+1)k - \frac12 k(k+1) = \frac12 k (2n-k+1)$である.

したがって,とり得る値の個数は,
$\frac12 k(2n-k+1) - \frac12 k(k+1) + 1 = \frac12 k(2n-2k)+1 = k(n-k)+1$個である.
