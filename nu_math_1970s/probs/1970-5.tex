\begin{problem}
  次の(1),(2)を証明せよ.
\begin{description}
\item[(1)]$\sin x$は,$x$の整式としては表わせない.
\item[(2)]$f(x)$は実数全体を定義域とする微分できる関数で,$f(1)=0$である.このとき
\[ g(x)=\begin{cases} \displaystyle \frac{f(x)}{x-1} \hspace{1zw} (x\neq1\text{のとき}) \\
f'(1) \hspace{1zw} (x=1\text{のとき}) \end{cases} \]
とおけば,$g(x)$は連続関数である.
\end{description}
\end{problem}

\begin{enumerate}
  \item $\sin x = \retuwa{k=0}{n}a_kx^k$と仮定する.($a_k$は任意の実数)

  このとき,$(\sin x)^4 = \sin x$であるので,$(\sin x)^{(4l)}=\sin x$\ ($l$は整数)となる.ここで,$4l > n$となる$l$を考えると,$(\sin x)^{(4l)} = \sin x  = 0$となるため,これは矛盾である.したがって,$\sin x$は$x$の整式で表すことは出来ない.

  \item 平均値の定理から,
  \[\frac{f(x) - f(1)}{x-1} = f'(c)\]
  を満たす$c$が存在する位置について,場合分けを行う.
  \begin{enumerate}[(i)]
    \item $x>1$のとき,$1 < c < x$の位置に存在する.
    したがって,$\dlim{x \to 1}\frac{f(x)-f(1)}{x-1} = f'(c)$となる.
    \item $x<1$のとき,$x < c < 1$の位置に存在する.
    したがって,$\dlim{x \to 1}\frac{f(x)-f(1)}{x-1} = f'(c)$となる.
  \end{enumerate}
  よって,(i),(ii)から,$\dlim{x \to 1}\frac{f(x)-f(1)}{x-1} = f'(c)$となるから,$g(x)$は連続関数である.
\end{enumerate}
