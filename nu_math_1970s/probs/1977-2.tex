
\begin{problem}
  3けたの素数$p$の百の位の数字を$a$,十の位の数字を$b$,一の位の数字を$c$とする.
このとき,2次方程式$ax^2+bx+c=0$は整数解をもたないことを証明せよ.
\end{problem}

条件式から,$\trenritu{%
  100a+10b+c=p \houteisiki{\label{19772-1}} \\
  ax^2+bx+c=0 \houteisiki{\label{19772-2}}%
}$
であり,$\eqref{19772-1}-\eqref{19772-2}$より,
$(10-x)(10+x)a + (10-x)b = p$\qquad $(10-x)\{(10+x)a+b\}=p$が導かれる.

これが素数となるためには,
\begin{enumerate}[(i)]
  \item $10-x =1$のとき,つまり,$(10+x)a+b=p$となるとき,$x=9, 19a+b=p$であり,$100a+10b+c=19a+b$となり,矛盾する.
  \item $10-x=p$のとき,つまり,$(10+x)a+b=1$となるとき,$(20-p)a+b=1$であり,\eqref{19772-1} より,$p=100a+10b+1 \geq 11$である.したがって,$20-p \leq -91$となり,$-91a + b \leq 0$となるから,矛盾する.
  \item $10-x=-1$のとき,つまり,$(10+x)a+b=-p$のとき,$x=11$であり,
  $10a+10b+c=-(21a+b)$となり,矛盾する.
  \item $10-x=-p$のとき,つまり,$(10+x)a+b=-1$のとき,$x = 10+p$であり,
  $(20+p)a+b \geq 0$となり,矛盾する.
\end{enumerate}

(i)~(iv)より,題意は示された.
\qed
