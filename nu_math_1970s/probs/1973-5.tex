\begin{problem}
関数$f(x)$ $(a \leqq x \leqq b)$が正の第2次導関数をもつとき,
曲線$y=f(x)$の上に点$P$をとって,
$P$における接線とこの曲線および2直線$x=a$,$x=b$とで囲まれた部分の面積を最小にするには,
点$P$をどのようにとればよいか.
\end{problem}

$f''(x) > 0$である.Pにおける接線は,
\begin{align*}
  y &= f'(p) (x-p) + f(p) \\
  &= \dint{a}{b}\zettaiti{f(x) - f'(p)(x-p) + f(p)}\,dx
\end{align*}
ここで,$f(x)$の部分は,pによらない定数となるから,
\[S(p) = \dint{a}{b}f'(p)(x-p)+f(p)\,dx\]
が最大となる$p$を求めればよい.
\begin{align*}
  S(p) &=
  \teisekibun{\bunsuu12 f'(p)x^2 + (f(p)-f'(p))\cdot p}{a}{b}\\
  &=\bunsuu12 f'(p)(b^2-a^2) + (f(p)-f'(p)\cdot p)(b-a)\\
  S'(p) &= \bunsuu12 f''(p)(b^2-a^2)+(f'(p)-f'(p)-f''(p)\cdot p)(b-a) \\
  &= \bunsuu12 f''(p) (b-a)\{(b+a)-2p\}
\end{align*}
ここで,$f''(p) > 0, (b-a) \geq 0$より,
\RESETKEYA
\setkeys{zogen}{
  hensu=p,
  ranaa=a,ranac=\frac{a+b}{2},ranae=b,
  ranbb=+,ranbc=0,ranbd=-,
  rancb=\nearrow,rancd=\searrow,
  kansub=S'(p),kansuc=S(p)
}
\zogen(3,5)

増減表より,$S(p)_{\mathrm{Max}}$となるのは,$p = \frac{a+b}{2}$のときである.\\したがって,求める点Pは$(x,y) =\left(\bunsuu{a+b}{2},f\left(\bunsuu{a+b}{2}\right)\right)$
\Ans{P$\left(\bunsuu{a+b}{2},f\left(\bunsuu{a+b}{2}\right)\right)$}
