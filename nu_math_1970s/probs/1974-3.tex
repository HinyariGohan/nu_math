\begin{problem}
  底辺$a$,高さ$h$の2等辺三角形がある.
\begin{enumerate}
\item この3角形の内接円の半径$r$を$a$と$h$を用いて表せ.
\item $n$が0でない整数で,$ah^n=1$を満たしながら$a$,$h$が変化するとき,
$\displaystyle\lim_{a\to\infty}\frac{r}{a}$を求めよ.
\end{enumerate}
\end{problem}



\begin{enumerate}
  \item
  \begin{mawarikomi}{}{
    \begin{pszahyou*}[ul=4mm](-0.5,7)(-0.5,10)
      \tenretu*{A(5,0)}
      \kandk\O{75}\A{105}\P
      \Suisen\P\O\A\H
      \Takakkei{\O\A\P}
      \Drawline{\P\H}
    \end{pszahyou*}
  }
  この三角形の面積を$S$とすると,
  \begin{align*}
    S &= \bunsuu12ah \\
    \bunsuu12ah &= r\left(a + 2\sqrt{h^2 + \bunsuu{a^2}4}\right) \\
    r &= \bunsuu{ah}{2\left(a + 2\sqrt{h^2 + \bunsuu{a^2}{4}}\right)}
  \end{align*}
\end{mawarikomi}

\Ans{$r = \bunsuu{ah}{2\left(a + 2\sqrt{h^2 + \bunsuu{a^2}{4}}\right)}$}

  \item $ah^n = 1$より,$h^n = \bunsuu1a$,したがって,$h = \left(\bunsuu1a\right)^{\frac1{n}}$

  \begin{align*}
    &\dumyeq
    \frac{ah}{2\left( a+2\sqrt{h^2 + \frac{a^2}{4}} \right)}
    =
    \frac{a\left( \frac{1}{a} \right)^{\frac{1}{n}}}
    {2\left( a + 2\sqrt{\left( \frac{1}{a} \right)^{2n} + \frac{a^2}{4}} \right)}
    =
    \frac{\left( \frac{1}{a} \right)^{\frac{1}{n}}}
    {2\left( 1 + 2\sqrt{\left( \frac{1}{a} \right)^{2n+2} + \frac14} \right)} \\
    &=
    \frac{\left( \frac{1}{a} \right)^{\frac{1}{n}}}
    {2\left\{ 1 - 4\left( \left( \frac{1}{a} \right)^{2n+2} + \frac14 \right) \right\}}
    = \frac{\left( \frac{1}{a} \right)^{\frac{1}{n}}}{-8\left( \frac{1}{a} \right)^{2n+2}}
    = -\frac18 \times \frac{\frac{1}{a^{\frac{1}{n}}}}{\frac{1}{a^{2n+2}}}\\
    &= -\frac18 a^{2n+2-\frac{1}{n}}
  \end{align*}

  ここで,$\frac{1}{n} = 2n +2$を解いて,$n = -1 \pm \sqrt3$であるから,
  \begin{enumerate}
    \item $-1-\sqrt3 < n -1 + \sqrt3$のとき,
    $\dlim{a\to\infty}-\frac18 a^{2n+2-\frac{1}{n}} = 0$

    \item $n = -1 \pm \sqrt3$のとき,
    $\dlim{a\to\infty}-\frac18 a^{2n+2-\frac{1}{n}} = -\frac18$

    \item $n < -1-\sqrt3$のとき,
    $\dlim{a\to\infty}-\frac18 a^{2n+2-\frac{1}{n}} = 0$

    \item $n > -1+\sqrt3$のとき,
    $\dlim{a\to\infty}-\frac18 a^{2n+2-\frac{1}{n}} = -\infty$
  \end{enumerate}

  したがって,(i)~(iv)より,

  \Ans{$\dlim{a\to\infty}\frac{r}{a} =
  \begin{emcases}
    0 & (n < -1-\sqrt3, -1-\sqrt3 < n < -1+\sqrt3 のとき)\\
    -\frac18 &(n = -1\pm\sqrt3 のとき) \\
    -\infty &(n > -1+\sqrt3 のとき)
  \end{emcases}
  $}
\end{enumerate}
