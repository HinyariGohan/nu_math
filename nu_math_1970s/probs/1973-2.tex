\begin{problem}
$n$を定まった正の整数とし,$1 \leqq k \leqq n$なる整数$k$のおのおのに,
$1 \leqq r \leqq n$なる整数$r$を対応させる関数$r=f(k)$があって,
$k_1<k_2$ならばつねに$f(k_1) \leqq f(k_2)$であるとする.
このとき,$f(m)=m$となる整数$m$が存在することを証明せよ.
\end{problem}

$r = f(k)$について,$k_1 < k_2$ならば,$f(k_1) \leq f(k_2)$より,$f(k)$は広義単調増加関数であるから,
\[\trenritu{f(1) \leq f(2) \leq \cdots \leq f(k) \leq \cdots \leq f(n) \\ 1 \leq \cdots \leq r \leq \cdots \leq n}\]
となる.ここで,$f(m) = m$となる$m$が存在しないと仮定する.

\begin{enumerate}[(i)]
  \item $m = 1$のとき,$f(1) \neq 1$より,$f(1) \geq 2$
  \item $m = 1,2,\cdots,l$のとき,$f(m) \neq m$が成立すると仮定すると,$f(l) \neq l$と,$f(l)\geq f(l-1)$より,$f(l) \geq l+1$となる.
\end{enumerate}

(i),(ii)から,$1, 2, \cdots ,n$の自然数$m$について,$f(m) \geq m+1$が成立する.

ここで,$1 \leq f(n) \leq n$より,$f(n) \geq n+1$となることはできないため,少なくとも1つは$f(m) = m$となる自然数$m$が存在する.\qed
