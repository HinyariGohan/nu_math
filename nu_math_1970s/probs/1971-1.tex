
\begin{problem}
  次のおのおのを証明せよ.
\begin{enumerate}
\item $n$ $(\geqq4)$個の点があって,
その中からどの4個の点をとってもそれらを通る円がかけるとき,
この$n$個の点は1つの円周上にある.
\item 平面上に4角形$S$があって,
$S$の任意の2点の距離が1をこえないならば,
$S$の面積は$\displaystyle\frac{1}{2}$をこえない.
\end{enumerate}
\end{problem}

\begin{enumerate}
  \item $h$個の点に,1から$n$までの番号を振り分ける.
  次の点1,2,3と点4~$n$のそれぞれを通る円について考える.
  ここで,任意の3点を通る円は1通りしか存在しないため,点4~$n$は
  それぞれ点1,2,3を通る1通りの円上に存在するため,この$n$個の点は
  1つの円上にあることが示された.
  \qed

  \item \begin{enumerate}
    \item すべての頂点の角の大きさが$180\degree$より小さいとき,
    \[\mathrm{OA} = a, \mathrm{OB} = b, \mathrm{OC} = c,
    \mathrm{OD} = d\]
    とする.四角形の面積を$S$とすると,
    \begin{align*}
      S &= \frac12 ab\sin\theta + \frac12 bc\sin\theta + \frac12 cd\sin\theta + \frac 12da\sin\theta \\
      &= \frac12(a+c)(b+d)\sin\theta \\
      &= \frac12 \zettaiti{AC}\zettaiti{BD} \sin\theta
    \end{align*}
    となる.

    ここで,任意の2点の距離が1を超えないことから,$S < \frac12\sin\theta$より,
    $S<\frac12$となる.

    \item ある角の大きさが$180\degree$より大きい四角形のとき,
    四角形の面積$S$は,$S< \frac12 \mathrm{AB}\cdot\mathrm{BD}\sin\kaku{B}$より,$S < \frac12 \sin\kaku{B}$となるから,$S < \frac12$である.
  \end{enumerate}

  よって,(i),(ii)から,すべての三角形に対して任意の2点の長さが1を超えないならば,$S$の面積は$\frac12$を超えない. \qed
\end{enumerate}
