
\begin{problem}
  平面上に3角形$ABC$が与えられている.
この平面上の点$P$に対して,
$AP$の中点を$Q$,$BQ$の中点を$R$,$CR$の中点を$S$とする.
このとき,$S=P$となる点$P$がただ1つ存在することを証明せよ.
また,この点を$P_0$とするとき,
$\triangle ABC$と$\triangle P_0BC$の面積の比を求めよ.
\end{problem}

$\Vec{AB} = \vec{b}, \Vec{AC} = \vec{c}$とする.また,$\Vec{AP} = s\vec{b} + t\vec{c}$とする.

このとき,$\trenritu{\Vec{AQ} = \frac{1}{2}(s\vec{b}+t\vec{c}) \\
\Vec{AR} = \frac{1}{4}(s\vec{b}+t\vec{c}) + \frac12 \vec{b} \\
\Vec{AP} = \Vec{AS} = \frac{1}{8}(s\vec{b} + t\vec{c}) + \frac14\vec{b} + \frac12 \vec{c}}$より,
それぞれ係数比較をして,
$\trenritu{\frac18 s + \frac14 = s \\ \frac18t + \frac12 = t}$である.
よってこれを$s,t$について解くと,
$(s,t) = \left( \frac27,\,\frac47 \right) $となるため,$\mathrm{S} = \mathrm{P}$となる点がただ一つ存在する.

直線ABとBCの交点を$\mathrm{P}_1$とする.
このとき,$k \bekutoru*[caprm]{AP_0}=\bekutoru*[caprm]{AP_1}$より,$k=\frac76$となるため,
$\bekutoru*[caprm]{AP_1} : \bekutoru*[caprm]{P_0P_1} = 7:1$となり,$\sankaku{ABC} : \sankaku*[caprm]{P_0BC} = 7:1$となる.
