\begin{problem}
$a$,$b$,$c$を複素数とするとき,次のことは正しいか.
正しいものは証明し,正しくないものについては反例(成り立たない例)をあげよ.
\begin{enumerate}
\item $ab$,$bc$,$ca$がすべて0ならば,$a$,$b$,$c$はすべて0である.
\item $a+b$,$b+c$,$c+a$がすべて実数ならば,$a$,$b$,$c$はすべて実数である.
\item $a^2+b^2+c^2=0$ならば$a$,$b$,$c$はすべて0である.
\item $a+b+c=0$,$ab+bc+ca=0$ならば,$a^3=b^3=c^3$である.
\end{enumerate}
\end{problem}

\begin{enumerate}
  \item $a=0,b=0,c=1$のとき,$ab,bc,ca$は0となるが,$a,b,c$は0ではないため,偽である.
  \item $a = c+fi, b=g+hi, c=j+ji$とする.$a+b, b+c, c+a$が実数であるから,$f+h=0,h+k=0,k+f=0$となり,これを満たす$f,h,k$の組み合わせは,$(f,h,k)=(0,0,0)$となる.

  よって,$a+b,b+c,c+a$がすべて実数であるとき,$a,b,c$は実数となる.
  \item $a^2 \geq 0, b^2 \geq 0, c^2 \geq 0$であるから,$a^2 + b^2 + c^2 = 0$となるには,$a^2 = 0, b^2 = 0, c^2 = 0$でなければならないから,$a = b = c = 0$となる.
  \item $abc = d$とすると,解と係数の関係から,$a,b,c$は$x^3-d = 0$の解となる.よって,$a=b=c$となる.
\end{enumerate}
