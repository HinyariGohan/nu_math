\begin{problem}
   放物線$y=x^2$上の異なる3点$(x_1, \, y_1)$,$(x_2, \, y_2)$,$(x_3, \, y_3)$における法線が1点で交わるとき,
$x_1+x_2+x_3=0$であることを証明せよ.
(曲線上の1点で,接線に垂直な直線を,その点における曲線の法線という)
\end{problem}

\ten{}{x_1}{y_1} における$y=x^2$の法線の方程式は,
\begin{align}
  y = -\bunsuu{1}{2x_1}(x-x_1) - {x_1}^2 \label{197131}
\end{align}
同様にして,\ten{}{x_2}{y_2} における場合は,
\begin{align}
  y = -\bunsuu{1}{2x_2}(x-x_2) + {x_2}^2 \label{197132}
\end{align}
\eqref{197131},\eqref{197132}を連立して,
\begin{align*}
  -\bunsuu{1}{2x_1}x + {x_1}^2 + \bunsuu12 &= -\bunsuu1{2x_2}x + {x_2}^2 + \bunsuu12\\
  -\bunsuu12\left(\bunsuu{1}{x_1} - \bunsuu{1}{x_2}\right)x &= {x_2}^2 - {x_1}^2\\
  x&=2({x_1}^2-{x_2}^2) \cdot \left(\bunsuu{x_1x_2}{x_2-x_1}\right)\\
  &=2(x_1-x_2)(x_1 + x_2) \cdot \bunsuu{x_1x_2}{x_2-x_1}\\
  &=-2x_1x_2(x_1+x_2)
\end{align*}

したがって,交点\ten{}{x}{y}$=$\ten{}{-2x_1x_2(x_1+x_2)}{{x_1}^2+x_1x_2+{x_2}^2+\bunsuu12}
である.

\ten{}{x_3}{y_3} における法線も同じ点で交わるから,
\begin{align*}
  {x_1}^2 + x_1x_2 + {x_2}^2 + \bunsuu12 &= -\bunsuu{1}{2x_3}\{-2x_1x_2(x_1+x_2)\} + {x_2}^2 + \bunsuu12 \\
  {x_1}^2 + x_1x_2 + {x_2}^2 &= \bunsuu{x_1x_2(x_1+x_2)}{x_3} + x_3^2\\
  0 &= x_3^3 - x_3({x_1}^2 + x_1x_2 + {x_2}^2) + x_1x_2(x_1+x_2) \\
  0 &= (x_3+x_2+x_1)(x_3-x_1)(x_3-x_2)
\end{align*}
ここで,$x_3 \neq x_1, x_3 \neq x_1$であるから,$x_3 + x_2 + x_1 = 0$である.
\hfill
(証明終了)
