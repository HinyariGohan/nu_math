\begin{problem}
下図のような立方体を考える.この立方体の8つの頂点の上を点Pが次の規則で移動する.\, 時刻0では点Pは頂点Aにいる.\, 時刻が1増えるごとに点Pは,今いる頂点と辺で結ばれている頂点に等確率で移動する.\, 例えば時刻$n$で点Pが頂点Hにいるとすると,時刻$n+1$まではそれぞれ$\bunsuu13$の確率で頂点D,E,Gのいずれかにいる.\, 自然数$n \geq 1$に対して,(i)点Pが時刻$n$までの間一度も頂点Aに戻らず,かつ時刻$n$で頂点B,D,Eのいずれかにいる確率を$p_n$,(ii)点Pが時刻$n$までの間一度も頂点Aに戻らず,かつ時刻$n$で頂点C,F,Hのいずれかにいる確率を$q_n$,(iii)点Pが時刻$n$までの間一度も頂点Aに戻らず,かつ時刻$n$で頂点Gにいる確率を$r_n$,とする.このとき,次の問に答えよ.
  \begin{enumerate}[(1)\ ]
    \item $p_2,q_2,r_2$と$p_3,q_3,r_3$を求めよ.
    \item $n \geq 2$のとき,$p_n,q_n,r_n$を求めよ.
    \item 自然数$m \geq 1$に対して,点Pが時刻$2m$で頂点Aに初めて戻る確率$s_m$を求めよ.
  \end{enumerate}
\end{problem}
